\documentclass{article}
\usepackage{lstlangbugs}

\begin{document}

\begin{lstlisting}[language=Stan]
/* 
comments
*/
# also a comment
// also a comment
data {
     // valid name
     int abcdefghijklmnopqrstuvwxyzABCDEFGHIJKLMNOPQRSTUVWXYZ0123456789_abc;
     // all types should be highlighed
     int a3;
     real foo[2];
     vector[3] bar;
     row_vector[3] baz;
     matrix[3,3] qux;
     simplex[3] quux;
     ordered[3] corge;
     positive_ordered[3] corge;
     corr_matrix[3] grault;
     cov_matrix[3] garply;
     
     // bounds
     int<lower=0>; bar1;
     int<upper=1>; bar2;
     int<lower=0, upper=2>; bar3;

     // bad names
     // includes .
     real foo.;
     // beings with number
     real 0foo;
     // begins with _
     real _foo;
}
transformed data {
     real xyzzy;
     int thud;
     row_vector grault2;
     matrix qux2;

     // all floating point literals should be recognized
     // all operators should be recognized
     // paren should be recognized;
     xyzzy <- 1234.5687 + .123 - (2.7e3 / 2E-5 * 135e-5);
     // integer literal
     thud <- -12309865;
     // ./ and .* should be recognized as operators
     grault2 <- grault .* garply ./ garply;
     // ' and \ should be regognized as operators
     qux2 <- qux' \ bar;
     
}
parameters {
    real fred;
    real plugh;
    
}
transformed parameters {    
}
model {
   // ~, <- are operators, 
   // T may be be recognized
   // normal is a function
   fred ~ normal(0, 1) T(-0.5, 0.5);
   // interior block
   { 
       real tmp;
       // for, in should be highlighted
       for (i in 1:10) {
	   tmp <- tmp + 0.1;
       }
   }
   // lp__ should be highlighted
   // normal_log as a function
   lp__ <- lp__ + normal_log(plugh, 0, 1);

   // print statement with string literal
   print("hello");

}
generated quantities {
   real foo1;
   foo1 <- foo + 1;
}
\end{lstlisting}

\begin{lstlisting}[language=JAGS]
model {
   for (i in 1:Num) {
      rt[i] ~ dbin(pt[i], nt[i]);
      rc[i] ~ dbin(pc[i], nc[i]);
      logit(pc[i]) <- mu[i] 
      logit(pt[i]) <- mu[i] + delta[i];
      delta[i] ~ dnorm(d, tau);
      mu[i] ~ dnorm(0.0, 1.0E-5);
   }
   d ~ dnorm(0.0, 1.0E-6);
   tau ~ dgamma(1.0E-3, 1.0E-3);
   delta.new ~ dnorm(d,tau);
   sigma <- 1/sqrt(tau);
}
\end{lstlisting}

\end{document}

